\section{Our Approach: Hierarchical Dirichlet  Clustering}
Describe the clustering procedure--refer to ISRR when needed.

\begin{algorithm}[t]
\caption{The Transition State Clustering Algorithm \todo{Fix with changes} \label{algotext}}
\begin{algorithmic}[1]
\scriptsize
\State \textsf{Input: } $\mathcal{D}$, $\rho$ pruning parameter, and $\delta$ compaction parameter.

\State $n(t) = \binom{\mathbf{x}(t+1)}{\mathbf{x}(t)}$. 

\State Cluster the vectors $n(t)$ using DP-GMM assigning each state to its most likely cluster. 

\State \emph{Transition states} are times when $n(t)$ is in a different cluster than $n(t+1)$. 

\State Remove states that transition to and from clusters with less than a fraction of $p$ demonstrations. 

\State Remove consecutive transition states when the L2 distance between these transitions is less than $\delta$. 

\State Cluster the remaining transition states in the state space $\mathbf{x}(t+1)$ using DP-GMM.

\State Within each state-space cluster, sub-cluster the transition states temporally. 

\State \textsf{Output: } A set $\mathcal{M}$ of clusters of transition states and the associated with each cluster a time interval of transition times.
\end{algorithmic}
\end{algorithm}

\noindent \todo{Changes to Algorithm}\\
-- Skill Weighted Pruning\\
-- Final Time Clustering over k-folds to regularization

