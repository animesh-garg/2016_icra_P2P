
\begin{algorithm}[t!]
\small
\DontPrintSemicolon
\caption{\textbf{\TSC:} Visual-Transition State Clustering}
\KwData{Set of demonstrations:$\mathcal{D}$, $\rho$ pruning parameter}
\KwResult{Set of Predicted Transitions  $\mathcal{T}_i,\enspace \forall \, d_i \in \mathcal{D}$}
% \Begin{
    \ForEach{$d_i \in \mathcal{D}$}{
        $z_i \gets$ VisualFeatures $(d_i, params)$\;
        $k_i \gets$ KinematicFeatures $(d_i, params)$\;
        %  {\scriptsize  \tcp*[l]{concatenate kinematic and visual features}}
        $\mathbf{x}_i(t) \gets \binom{\mathbf{k_i}(t+1)}{\mathbf{z_i}(t)}\ \forall t \in \{1,\ldots, T_i\}$\;
        $\bar{x}_i(t) \gets \big[\, \mathbf{x}(t+1)^T, \ \mathbf{x}(t)^T, \ \mathbf{x}(t-1)^T \,\big]^T, \ \forall t$\;
        $N \gets \big[N^T, \bar{x}_i(1)^T,\ldots,\bar{x}_i(T_i)^T\big]^T$\;
    }
    {\scriptsize  \tcp*[l]{Cluster to get Change Points}}
    $\mathcal{C}^{CP} = \textnormal{DPGMM}(N, \alpha_0)$ {\scriptsize  \tcp*[r]{$\alpha_0$ is hyperparameter}}
    \ForEach{$N(n)\in\mathcal{C}^{CP}_{i},\ N(n+1)\in\mathcal{C}^{CP}_{j},i\neq j$ } {$CP \gets CP \cup \{N(n)\}$}
    {\scriptsize  \tcp*[l]{Cluster over Kinematic Feature Subspace}}
    $\mathcal{C}_{1} = \textnormal{DPGMM}(CP, \alpha_1)$ {\scriptsize  \tcp*[r]{$\mathcal{C}_{1}$: set of clusters}}
    %  \ForEach{$C^{1}_k \in \mathcal{C}^{1}$}{
    %     \uIf{$\sum_{d_i} \mathbbm{1}\Big( \sum_{n:N(n)\in d_i} \mathbbm{1}( CP(n) \in C^{1}_k) \geq 1 \Big) \geq \rho |\mathcal{D}|$}{
    %     $\mathcal{C}^{1}\gets \mathcal{C}^{1}\setminus \mathcal{C}^{1}_k$ {\scriptsize  \tcp*[r]{Cluster Pruning}}
    %     }
    %  }
    \ForEach{$C_{k} \in \mathcal{C}_{1}$}{
        $CP(C_{k}) \gets \{CP(n) \in C_{k}, \forall n \in \{1,\ldots,|CP|\}\}$\;
        {\scriptsize  \tcp*[l]{Cluster over Visual Feature Subspace}}
        $\mathcal{C}_{k2} \gets \textnormal{DPGMM}(CP(C_{k}), \alpha_2)$\;
        \ForEach{$C_{kk'} \in \mathcal{C}_{k2}$}{
            \If{$\sum_{d_i} \mathbbm{1}\Big( \sum_{n:N(n)\in d_i} \mathbbm{1}( CP(n) \in C_{kk'}) \geq 1 \Big) \leq \rho |\mathcal{D}|$}
            {$\mathcal{C}_{k2}\gets \mathcal{C}_{k2}\setminus \{C_{kk'}\}$ {\scriptsize \tcp*[r]{Cluster Pruning}}
            }
            {\scriptsize \tcp*[l]{collect intra-cluster transitions}}
            $T_i \gets T_i \cup \{CP(n) \in C^{kk'}, \forall n:N(n)\in d_i\}$\;
        }
    }
    {\scriptsize  \tcp*[l]{Cluster over time to predict Transition Windows}}
    \ForEach{$d_i \in \mathcal{D}$}{
        Repeat steps 1-17 for $\mathcal{D}' = \mathcal{D} \setminus d_i$\;
        $\mathtt{T}_j \gets \mathtt{T}_j \cup T_j^{(i)}, \ \{\forall\ j: d_j \in \mathcal{D}'\}${\scriptsize \tcp*[r]{$T_j^{(i)}$: \textit{i}th iteration}} 
        % $\mathcal{T}_i \gets DPGMM (\mathtt{T}, \alpha_4)${\scriptsize  \tcp*[r]{store cluster $\mu$ and $\Sigma$}}
    }
    \lForEach{$d_i \in \mathcal{D}$}{$\mathcal{T}_i \gets DPGMM (\mathtt{T}_i, \alpha_4)$}
    \nl\KwRet{$\mathcal{T}_i,\enspace \forall \, d_i \in \mathcal{D}$}
% }
\label{alg:vtsc}
% \vspace*{-10pt}
\end{algorithm}

\section{Visual Transition State Clustering}
We present the \tsc algorithm in Algorithm~\ref{alg:vtsc}.
In this section, we overview the extensions to the previously proposed TSC algorithm~\cite{krishnan2015tsc}.

\subsection{Visual Features}
The first step in our extension is to define an augmented state space $\mathbf{x}(t) = \binom{k(t)}{z(t)}$, where $k(t)$ are the kinematic features and $z(t)$ are the visual features.
We use layers from a pre-trained Convolutional Neural Network (CNNs) to derive the features frame-by-frame.
CNNs are increasingly popular for image classification and as a result a number of image classification CNNs exist which are trained on terabytes of natural images.
Intuitively, CNNs are trained to classify based on aggregations (pools) of hierarchical convolutions of the pixels.
Removing the aggregations and the classifiers, results in convolutional filters which can be used to derive features.

However, we found that to use these features required a number of pre-processing and post-processing steps; in addition to a number of design choices within the CNN itself like what layer in the convolutional hierarchy to use.

\subsubsection{Pre-processing}
As the image classification CNNs are trained on static images, their features are optimized for identifying salient edges and colors.
Since these features are not temporal, they do not differentiate between robot and workspace features.
Furthermore, since we aggregate across demonstrations, we need to ensure that these features are largely consistent.
To avoid variance due to extraneous objects and small changes in lighting, we first pre-processed the videos.

We first cropped all videos by equal amounts to capture only the workspace where most of the robot manipulation happened. 
Then, they were rescaled to 640 x 480 dimensions with normalization to have a zero mean RGB value as standard \cite{krizhevsky2012imagenet,simonyan2014very}. 
Finally, for computational reasons, the videos were downsampled to \todo{x} frame per second.  
All of pre-processing happened with the open source program \texttt{ffmpeg}.

\subsubsection{Visual Featurization}
Once the images were pre-processed, we applied the convolutional filters from the pre-trained neural networks. In particular, we explore two architectures.

\vspace{0.25em}
\noindent\textbf{AlexNet: } Krizhevsky et al. proposed a convolutional neural network architecture for image classification \cite{krizhevsky2012imagenet}. This architecture was seminal as it proposed multiple stacked convolutional filters (5 in all).   

\vspace{0.25em}
\noindent\textbf{VGG: } Simoyan et al. proposed an alternative architecture termed VGG which increased the number of convolutional layers significantly (16 in all) \cite{simonyan2014very}.

\vspace{0.25em}

In this work, we explore selecting convolutional features from different layers in the hierarchy. 
This problem is called transfer learning and has been well studied \todo{cite}.
It is established that earlier layers in the hierarchy give more general features while later layers give more specific ones.
In our experiments, we explore the level of generality of features required for segmentation. 
We also compare these features to other visual featurization techniques such as SIFT and SURF.

\subsubsection{Post-Processing: Encoding}
After constructing these features, the next step is encoding the results of the convolutional filter into a vector $z(t)$.
We explore three encoding techniques: (1) raw values, (2) VLAD, and (3) LCD.

\vspace{0.25em}
\noindent\textbf{Raw Filter Values: } The first encoding technique that we explore is using the raw convolutional filter values. We stack these values into a vector which constructs $z(t)$.

\vspace{0.25em}
\noindent\textbf{Vector of Locally Aggregated Descriptors (VLAD)\cite{arandjelovic2013all}: }
Vector of Locally Aggregated Descriptors (VLAD) image encoding as proposed by is a feature encoding and pooling method. 
It transforms an incoming variable-size set of independent samples into a fixed size vector representation.
VLAD encodes a set of local feature descriptors $I=\{x_1,\ldots,x_n\}$ extracted from an image using a code book $\mathrm{C} = \{c_1, \ldots, c_K \}$ built using a clustering method. With K-coarse cluster centers, generated by K-means, we can obtain the difference vector for every center $c_j$ as:\vspace{-5pt}
\[ u_k = \sum_{i:\,1\text{-}nbd(x_i)=\{c_j\}} (x_i- c_j)
\]
where $k\text{-}nbd(x_i)$ indicates k(=1) nearest neighbors of $x_i$ among K coarse centers.
VLAD vector is obtained by concatenating $u_k$ over all the K centers, $V = [u_1^T, \ldots, u_K^T]$, with $V$ of size K$D$ where $D$ is the dimension of incoming points $x_i$. 

\vspace{0.25em}
\noindent\textbf{Latent Concept Descriptors (LCD) \cite{xu2014discriminative}: }
Convolutional layers contain important spatial information, however at the cost of high dimensional representation. For instance the feature dimension at a convolutional layer can be $c\times c \times M$, where $c$ is filter size and $M$ is number of filters. However using the latent concept descriptors, a layer of size $c\times c \times M$ can be converted into $c^2$ latent concept descriptors with M dimensions. Each latent concept descriptor represents the responses from the M filters for a specific pooling location. 

\subsubsection{Post-Processing: Dimensionality Reduction}
After encoding, we feed the features $z(t)$ through a dimensionality reduction process.
This is for computational reasons as the visual features are very high dimensional and the Transition State Clustering algorithm suffers from the curse of dimensionality.
Furthermore, almost all GMM-based clustering algorithms converge to a local minima and very high dimensional feature spaces can lead to numerical instability or inconsistent behavior.
Principal Component Analysis (PCA) is widely applied in computer vision. 
We explore alternative dimensionality reduction techniques to find desirable properties of the dimensionality reduction that result in improve segmentation performance.
In particular, we explore Gaussian Random Projections (GRP), and Canonical Correlation Analysis (CCA).

\subsection{Other Extensions}
\subsubsection{Sliding Window States}
In our prior work, we used the concatenation of kinematic and visual features at a given time $t$ as the state representation $\binom{k(t)}{z(t)}$. However, the use of rolling time window in both kinematics and visual space allows capturing dynamics. 
We empirically found that the use of $[k(t-1), k(t), k(t+1)]^T$ as the kinematic state representation led to improved segmentation accuracy. Other lengths of temporal history were experimented with as shown in Figure~\todo{add figure}.

\subsubsection{Skill-Weighted Pruning} 
Different demonstrators often have different skill levels, and we extend our prior work to prune transition states in a weighted manner.
Let, $w_i$ be the weight for each demonstration $d_i \in \mathcal{D}$, such that $w_i \in [0,1]$ and $\hat{w}_i = \frac{w_i}{\sum w_i}$. Then a cluster $C_{kk'}$ is pruned if it does not contain points from at least $\rho$ fraction of demonstrations. This converts to:
\[\sum_{d_i} \hat{w}_i\mathbbm{1}\Big( \sum_{n:N(n)\in d_i} \mathbbm{1}( CP(n) \in C_{kk'}) \geq 1 \Big) \leq \rho 
\]


\subsubsection{Robust Temporal Clustering}
In our prior work, we used a hierarchical application of DP-GMM first clustering spatially and then temporally. In this work, we explore a more robust variant of this approach.
We iteratively hold out one out of the $N$ demonstrations and apply \tsc to the remaining demonstrations.
Then, over $N-1$ runs of \tsc, $N-1$ predictions are made for each $d_i \in \mathcal{D}$. We temporally cluster the transitions across $N-1$ predictions, to estimate final transition state centroid and variance. This step is illustrated in step 20-21 of Algorithm~\ref{alg:v.tsc}.
% the centroids of the transition state clusters.

% \vspace{0.5em}