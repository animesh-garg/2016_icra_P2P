\documentclass[letterpaper, 10 pt, conference]{ieeeconf}

%\let\labelindent\relax
\input{preamble}
\usepackage{blindtext}

%===============================================================
\title{\LARGE \bf
Pixels to Primitives: Learning Sub-Task Level Semantic Segmentation \\
of Multi-Step Task Trajectories from Video with Deep Learning }
% Pixels to Primitives (P2P)

\author{%
Adithyavairavan Murali*, Animesh Garg*, Sanjay Krishnan*, Florian Pokorny,\\ 
Pieter Abbeel, Trevor Darrell, Ken Goldberg \quad {\textcolor{blue}{[v0.1, \today\,\currenttime]}}
\thanks{\hrule \vspace{5pt} * The authors contributed equally to the paper}%
\thanks{EECS \& IEOR, University of California, Berkeley CA USA; \texttt{\{adithya\_murali, animesh.garg, sanjaykrishnan, ftpokorny, pabbeel, trevor, goldberg\}@berkeley.edu}}%
% \thanks{$^{1}$EECS, University of California, Berkeley; {\{sanjaykrishnan, adithya\_murali\}@berkeley.edu}}%
% \thanks{$^{2}$IEOR and EECS, University of California, Berkeley; {\{animesh.garg, goldberg\}@berkeley.edu}}%
}
\IEEEoverridecommandlockouts %to enable thanks to appear
\newcommand{\sys}{\textsf{TSC+VIS}\xspace}

\begin{document}

\maketitle

\begin{abstract}
Segmentation is an important first step in analyzing long running robotic tasks. For reliable results, it is important to consider both visual and kinematic data, as visual data provides important information about the state of the workspace. 
Existing unsupervised segmentation methodologies are limited in the ways they can leverage visual data and rely on annotations or complete knowledge of all objects in the world. In this paper, we propose a framework that takes a step towards unsupervised segmentation of robotic demonstrations using raw video (i.e., pixel data). 
We identify key transition events in kinematic data, cluster transitions together using visual data, and then identify segments of the raw video corresponding to these clusters of transition events. 
The resulting video segments can be used to design error recovery actions, parameter tuning, action classification, 
and operator skill assessment. 
\todo{Our results on x suggest y}
\end{abstract} 

\documentclass[0-main.tex]{subfiles}
\begin{document}


\section{Introduction}
%A large and ever increasing number of surgical procedures are performed or aided by surgical robots such as Intuitive Surgical's \textit{da Vinci}.
The prevalence of robot-assisted surgery is leading to a proliferation of datasets consisting of kinematic and fixed-camera video recordings of surgical procedures.
While these datasets have the potential to facilitate learning and autonomy, the variability of surgical data poses a unique challenge.
By nature, surgical robots interact with deformable environments and varying anatomy. 
Therefore, one of the key challenges in utilizing these datasets is to learn a common sequential structure shared across multiple instances of the same surgical task.
One model for this sequential task structure is the Transition State model~\cite{krishnan2015tsc}, which identifies spatially and temporally similar changes in motion.
Thus, while the apparent data might be highly variable, the model often learns latent similarities in tool position or kinematic state.
The learned model temporally \emph{segments} trajectories into meaningful contiguous sections, which can ultimately facilitate local learning from demonstrations, skill assessment, and salvaging good local segments from otherwise inconsistent demonstrations.

In addition to the Transition State model, there are several recent proposals to learn segmentation criteria directly from data with minimal supervision~\cite{calinon2010learning, Niekum2015learning}.
Inherently, the success of these approaches is dependent on the state representation, which is particularly challenging for visual features; especially when we want these features to perform well across different tasks. 
Inspired by the recent success of deep neural networks in reinforcement learning~\cite{levine2015end, lenz2015deep}, this paper explores how segmentation can be learned from visual state representations extracted from \textit{deep} convolutional neural networks (CNNs).

\begin{figure}[t!]
\centering
\vspace{-5pt}
% \includegraphics[width=0.5\linewidth]{figures/insert}
\includegraphics[width=\linewidth]{figures/suturing-teaser-v4.png}
\caption{We apply \tsc to a suturing task. Each ``throw" of suturing repeats between four steps, and figure illustrates that \tsc extracts a segmentation that closely aligns with the manual annotation without supervision.}
\figlabel{surgical-teaser}
% \label{fig:pr2_toyplane}
\vspace{-15pt} 
\end{figure}
We propose Transition State Clustering with Deep Learning (\tsc), which extends our previous work~\cite{krishnan2015tsc} with automatically constructed visual features using pre-trained CNNs.
In our prior work, we identified changes in local linearity in each trajectory, and learned a model to infer regions of the state-space at which changes occurred.
We modeled these regions as generated from a hierarchical nonparametric Bayesian model, where the number of regions are determined by a Dirichlet Process and the shape of the regions are determined by a mixture of multivariate Gaussian random variables.
Our surprising finding is that while the original Transition State model was motivated for spatial states, it empirically performs well (in comparison to ground truth) even when applied to trajectories of visual filters derived from layers of CNNs.

However, the integration of the convolutional filters with kinematics is not as straight-forward as concatenating the two parameter spaces.
This paper addresses two new issues: (1) defining metrics across multiple sensing modalities, and (2) task-trajectory correspondence.
For (1), it can be challenging to define similarity metrics across states that have both spatial and visual components.
We propose a hierarchical clustering approach that first clusters over visual space and then over kinematics, which allows us to compare kinematics-kinematics and visual-visual.  
For (2), \tsc learns a segmentation model for a task given observation of trajectories of the task.
The problem is that we may want to know how to correspond states from the individual trajectories to the segments of the task learned across all trajectories.
To learn this correspondence, we propose a novel temporal clustering technique which leverages the Jackknife estimator to correspond global segmentation structure to temporal points in the individual trajectories.




 \iffalse
\begin{SCfigure*}[][t!]
    \centering
    \vspace{-10pt}
    \includegraphics[width=1.55\linewidth]{figures/sysArch}
    \caption{We use a visual processing pipeline with deep features to construct a trajectory of high-dimensional visual states $z(t)$.
    We concatenate encoded versions of these features with kinematics and apply hierarchical clustering to find segments.}%dont add more lines--seems to mess up formatting
    \label{fig:pipeline}
    \vspace{-10pt}
\end{SCfigure*}
\fi
% \fi

% \begin{figure*}[!t]
% \centering
% \includegraphics[width=0.8\linewidth]{figures/sysArch}
% \caption{\tsc is a task segmentation pipeline that starts with multimodal demonstrations (kinematics and vision), uses deep learning to featurize the frames of the video, and then clusters transition events to find segments.}
% \label{fig:pipeline}
% \vspace{-15pt}
% \end{figure*}


We evaluate \tsc on three datasets: (1) a synthetic 4 segment example, (2) JIGSAWS surgical needle passing,  and (3) JIGSAWS surgical suturing.
On the synthetic example, we find that \tsc recovers the 4 underlying segments in the presence of partial state observation (one kinematic state hidden), control noise, and sensor noise. 
We also compare \tsc with manual annotations when available.
On real datasets, we find that \tsc matches the manual annotation with up to 0.806 normalized mutual information.
Our results also suggest that including kinematics and vision results in increases of up-to 0.215 NMI over kinematics alone.
We demonstrate the benefits of using an unsupervised approach by presenting examples where \tsc discovers unlabeled segments due to human annotator error (as shown in Figure \ref{fig:suturing}), and can learn across demonstrations with widely varying operator skill levels (as shown in Table \ref{tab:jigsaws}).


%Clustering allows focussed LFD on each segment resulting in a better fit. 
%Clustering can identify inconsistent outliers and dirty data segments that cann corrupt LFD, also 
%Can identify segments where more consistent data is needed.
%can salvage good segments form bad demo sequences. 

%emph unsupervised
%avoids human bias
%apply consistent models -- can discover subtle transitions
%faster than humans 
  
\iffalse
\begin{figure}[ht]
\centering
\includegraphics[width=\columnwidth]{figures/architecture.pdf}
\caption{\todo{name} architecture. We use a pre-trained CNN to featurize raw video data for use in segmentation. After featurization, we combine the data with kinematic data and apply a Transition State Clustering algorithm to identify segments.}
\figlabel{arch}
\vspace{-1em}
\end{figure}
\fi

\end{document}
\documentclass[0-main.tex]{subfiles}
\begin{document}

\section{Related Work}

%\subsection{Related Work}

\subsection{Learning From Demonstrations}
One model for learning from demonstrations uses segmentation to discretize action spaces (skill-learning) which allows for efficient learning of complex tasks~\cite{ijspreet2002learning,pastor2009learning}.
This line of work largely focuses on pre-defined primitives.
Niekum et al. \cite{niekum2012learning} proposed an unsupervised extension to the motion primitive model by learning a set of primitives using the Beta-Process Autoregressive Hidden Markov Model (BP-AR-HMM).
The work by Niekum et al. does incorporate visual information, however, it does not use visual information to actually find segments.
Post segmentation,  Niekum et al. uses AR markers to estimate poses of all of the objects in the workspace.
The segments, discovered with kinematics alone, are then specified in each objects reference frame.
When the objects are then moved, the trajectory can be transferred using a Dynamic Motion Primitive model.

Calinon et al.~\cite{calinon2014skills, calinon2010evaluation} characterizes segments from demonstrations as skills that can be used to parametrize imitation learning.
In this line work, the authors apply Gaussian Mixture Models (GMMs) to cluster observations from the same mixture component.
A number of other works have leveraged this model for segmentation e.g., \cite{konidaris2009efficient, konidaris2011robot, subramanian2011learning}.
As we will later describe, Gaussian Mixture Models have a duality with switched linear dynamical systems \cite{moldovan2013dirichlet}.
Calinon et al. \cite{calinon2010evaluation} uses segmentation to teach a robot how to hit a moving ball.
They use visual features through a visual trajectory tracking of a ball.
The visual sensing model in Calinon et al. is tailored to the ball task, and in this paper, we use a set of general visual features for all tasks.

\subsection{Surgical Robotics}
%Surgical robotics has a number of challenges that make segmentation difficult \cite{krishnan2015tsc}.
%In Krishnan et al. \cite{krishnan2015tsc}, we found that hand-annotated visual features could significantly improve segmentation accuracy.
Other surgical robotics works have largely studied the problem of supervised segmentation using either segmented examples or a pre-defined dictionary of motions (similar to motion primitives) \cite{varadarajan2009data,tao2013surgical,lea15improved, zappella2013surgical, quellec2014segmentation}.


\subsection{Visual Gesture Recognition}
A number of recent works, attempt to segment human motions from videos \cite{hoai2011joint, tang2012learning, yang2013discovering, jones2014unsupervised, wu2014leveraging, wu2015watch}.
Tang et al. and Hoai et al. proposed supervised models for human action segmentation from video.
Building on the supervised models, there are a few unsuperivsed models for segmentation of human actions: Jones et al.\cite{jones2014unsupervised}, Yang et al. \cite{yang2013discovering}, Di Wu et al. \cite{wu2014leveraging} , and Chenxia Wu et al. \cite{wu2015watch}.
Jones et al. \cite{jones2014unsupervised} restricts their segmentation to learning from two views of the dataset (i.e., two demonstrations).
Yang et al. \cite{yang2013discovering} and Wu et al.  \cite{wu2015watch} use k-means to learn a dictionary of primitive motions, however, in prior work, we found that transition state clustering outperforms a standard k-means segmentation approach.
In fact, the model that we propose is complementary to these works and would be a robust drop-in-replacement for the k-means dictionary learning step \cite{krishnan2015tsc}.
The approach taken by Di Wu et al. is to parametrize human actions using a skeleton model, and they learn the parameters to this skeleton model using a deep neural network.
In this work, we explore using generic deep visual features for robotic segmentation without requiring task-specific optimization such as skeleton or action models using in human action recognition.

\subsection{Deep Features in Robotics}
Robotics is increasingly using deep features for visual sensing. For example, Lenz et al. uses pre-trained neural networks for object detection in grasping \cite{lenz2015deep} and
Levine et al. \cite{levine2015end} fine-tune pre-trained CNNs for policy learning.
For this reason, we decide to explore methodologies for using deep features in segmentation as well. We believe that segmentation is an important first step in a number of robot learning applications, and the appropriate choice of visual features is key to accurate segmentation.
We present an initial exploration of different visual featurization strategies and segmentation accuracy.

\iffalse
\subsection{Trajectory Segmentation Models}
Many unsupervised segmentation models either implicitly or explicitly model the dynamics as locally linear.
It is important to note that locally linear dynamics does not imply linear motions, as spiraling motions can be represented as linear systems. 
In \cite{elhamifar2009sparse}, videos are modeled as transitions on a lower-dimensional linear subspace and segments are defined as changes in these subspaces.
Willsky et al~\cite{willsky2009sharing} proposed BP-AR-HMM, which was applied by Niekum et al. in robotics \cite{niekum2012learning}.
This model is explicitly linear by fitting a autoregressive model to time-series. 
The linear function switches according to an HMM with states parametrized by a Beta-Bernoulli model (i.e., Beta Process). 
In fact, even the works that apply Gaussian Mixture Models for segmentation \cite{calinon2010learning, lee2015autonomous, kruger2012imitation}, implicitly fit a locally linear dynamical model.
Moldovan et al. \cite{moldovan2013dirichlet} proves that a Mixture of Gaussians model is equivalent to Bayesian Linear Regression; i.e., when applied to a time window it fits a linear transition between the states.
\fi

\end{document}
\section{Prior Work: Transition State Clustering}

We first overview the Transition State Clustering algorithm from our prior work \cite{krishnan2015tsc}.

\subsection{Transition State Clustering Overview}
The Transition State Clustering algorithm (\sys), learns clusters of states that mark dynamical regime transitions.

\subsubsection{Learning Transition States}
Let $\mathcal{D}=\{d_i\}$ be the set of demonstrations where each $d_i$ is a trajectory of fully observed robot states and each state is a vector in $\mathbb{R}^d$.
We model each demonstration as a switched linear dynamical system.
There is a finite set of $d \times d$ matrices $\{A_1,...,A_k\}$, and an i.i.d zero-mean additive Gaussian Markovian noise process $W(t)$ which accounts for noise in the dynamical model:
\[
\mathbf{x}(t+1) = A_{i}\mathbf{x}(t) + W(t) \text{ : } A_i \in \{A_1,...,A_k\}
\]
In this model, transitions between regimes are instantaneous where each time $t$ is associated with exactly one dynamical system matrix $1,...,k$.
\emph{Transition states} are defined as the last states before a dynamical regime transition in each demonstration.
Therefore, there will be times $t$ at which $A(t) \ne A(t+1)$.
A transition state is the state $x(t)$ at time $t$.

Suppose there was only one regime, then this would be a linear regression problem:
\[
\arg\min_A \|A X_t - X_{t+1}\|
\]
where $X_t$ is a matrix where each column vector is $x(t)$, and $X_{t+1}$ is a matrix where each column vector is the corresponding $x(t+1)$.
Moldovan et al. \cite{moldovan2013dirichlet} proves that fitting a Jointly Gaussian model to $n(t) = \binom{\mathbf{x}(t+1)}{\mathbf{x}(t)}$ is equivalent to Bayesian Linear Regression.
We use Dirichlet Process Gaussian Mixture Models (DP-GMM) to learn the regimes without have to set the number of regimes in advance.
Each cluster learned signifies a different regime, and co-linear states are in the same cluster.
To find transition states, we move along a trajectory from $t=1,...,t_f$, and find states at which $n(t)$ is in a different cluster than $n(t+1)$.
These points mark a transition between clusters (i.e., transition regimes).

\subsubsection{Learning Transition State Clusters}
A \emph{transition state cluster} is
defined as a clustering of the set of transition states across all demonstrations; partitioning these transition states into $m$ non-overlapping similar groups:
$
\mathcal{C} = \{C_1, C_2,...,C_m\}
$
In our prior work, we formalized the conditions under which meaningful clusters can be learned (i.e., consistency of demonstrations \cite{krishnan2015tsc}).
If we model the states at transition states as drawn from a GMM model: ${x}(t) \sim N(\mu_i, \Sigma_i)$.
Then, we can apply the DP-GMM again to cluster the state vectors at the transition states.
Each cluster defines an ellipsoidal region of the state-space space.

Each of these clusters will have constituent vectors where each $n(t)$ belongs to a demonstration $d_i$. 
Clusters whose constituent vectors come from fewer than a fraction $\rho$ demonstrations are \emph{pruned}.
$\rho$ should be set based on the expected rarity of outliers.

\vspace{0.5em}

\noindent \textbf{Given a consistent set of demonstrations, the algorithm finds a sequence of transition state clusters reached by at least a fraction $\rho$ of the demonstrations.}

\documentclass[0-main.tex]{subfiles}
\begin{document}

\section{Transition State Clustering \\ With Deep Learning}
\label{sec:vtsc}

As defined in Sec~\ref{sec:model}, a transition state is the last state before a dynamical regime transition. Such switching events can be modelled as a latent function of the current state $S:\mathcal{X}\mapsto \{0,1\}$, and we have noisy observations of switching events $\hat{S}(\mathbf{x}(t)) = S(\mathbf{x}(t)+Q(t))$, where $Q(t)$ is i.i.d measurement noise.

\noindent \textbf{Identifying Transitions: }
Suppose there was only one regime, then following from the Gaussian assumption, this would be a linear regression problem:
\[
\arg\min_A \|A X_t - X_{t+1}\|
\]
where $X_t$ is a matrix where each column vector is $\mathbf{x}(t)$.
If there are multiple regimes, Moldovan et al.~\cite{moldovan2013dirichlet} proves that fitting a Jointly Gaussian model to $n(t) = \binom{\mathbf{x}(t+1)}{\mathbf{x}(t)}$ is equivalent to Bayesian Linear Regression--and thus fitting a GMM finds locally linear regimes.
This GMM model gives us the switching events.
The observed switching events induce a common probability density $f$ over the state space $\mathcal{X}$ across all demonstrations. The density $f$ can be learned through a Gaussian Mixture Model parameter inference.

\subsection{Task Representation as a Sequence of Transition Clusters}
The task of learning an ordered set of transition state clusters $\mathcal{C}$ can be addressed by fitting a Non-parametric Gaussian Mixture Model to the set of demonstrations. 

\vspace{2pt}
\noindent \textbf{Multi-modal State Representation}: A Gaussian mixture model in a Euclidean space $\mathbb{R}^d$ assumes an $L_2$ metric. However in state representations composed of multiple sensing modalities such as kinematics and visual state, the $L_2$ metric may not be used naively. The problem is further exacerbated if the dimensions of different modalities are not comparable.

We address this problem by constructing a hierarchy of GMM clusters, starting with a subset of state representation, only one sensing modality say visual features. 
Then, we partition the dataset by each transitions most likely mixture.  
Within each partition, we fit GMM corresponding to a subsequent modality, say kinematics, and repeating this process for different modalities.

% \subsection{Visual Features}
\vspace{2pt}
\noindent\textbf{Visual Features}:
The Transition State Clustering with Deep Learning (\TSC) utilizes 
domain independent visual features from pre-trained CNNs. 
CNNs are increasingly popular for image classification and with existing models trained on millions of natural images.
Intuitively, CNNs classify based on aggregations (pools) of hierarchical convolutions of the pixels. Yosinski et al. noted that CNNs trained on natural images exhibit roughly the same Gabor filters and color blobs on the first layer~\cite{yosinski2014NIPS}. They established that earlier layers in the hierarchy give more general features while later layers give more specific ones. Hence, removing the aggregations and the classification layers results in convolutional filters which can be used to derive generic features across datasets.

We use layers from a pre-trained Convolutional Neural Network (CNNs) to derive the features frame-by-frame. In particular, we explore two architectures designed for image classification task on natural images: (a) \textbf{AlexNet: } Krizhevsky et al. proposed multilayer (5 in all) a CNN architecture \cite{krizhevsky2012imagenet}, and (b)~\textbf{VGG: } Simoyan et al. proposed an alternative architecture termed VGG (acronym for Visual Geometry Group) which increased the number of convolutional layers significantly (16 in all)~\cite{simonyan2014very}. In our experiments, we explore the level of generality of features required for segmentation. We also compare these features to other visual featurization techniques such as SIFT for the purpose of task segmentation using \TSC.


\vspace{2pt}
\noindent\textbf{Algorithm Overview}: \\
We define an augmented state space $\mathbf{x}(t) = \binom{k(t)}{z(t)}$, where $k(t) \in \mathbb{R}^k$ are the kinematic features and $z(t) \in \mathbb{R}^v$ are the visual features.
The augmented state for each demonstration $d_i \in \mathcal{D}$ is collected in a state vector $N$. GMM clustering over the sequence of states in $N$, results in the identification of the change points, or the switching events where $A(t) \ne A(t+1)$.

The change points from all demonstrations in  $\mathcal{D}$ are collected into a set $\mathcal{C}^{CP}$. Further hierarchical clustering uses state representations only at change points. 
Intuitively, the change point identification results in an over segmentation of the trajectory in state space, while subsequent clustering steps retain only a small subset of transition states (change points) that are consistent across the data set. 

Thereafter, we cluster in sub-spaces of each of the modalities -- perception and kinematics. We start with visual feature being 
% We extend our prior work with states defined with visual features, and present the \tsc algorithm in Algorithm~\ref{alg:vtsc}.
Within each visual feature space cluster, we model the kinematics change pointsto be drawn from a GMM:
$k \sim N(\mu_i, \sigma_i)$, 
then we can apply a GMM to the kinematic subspace of the change points. Finally, we perform a consistency check in recovered transition state clusters by pruning clusters which do not have change points from at least a $\rho$-fraction of the dataset.



\begin{algorithm}[t]
\small
\DontPrintSemicolon
\caption{\textbf{\TSC:} Transition State Clustering with Deep Learning \label{alg:vtsc}}
\KwData{Set of demonstrations:$\mathcal{D}$}
\SetKwInput{params}{Parameters}
\params{pruning factor ($\rho$), time window ($w$), PCA dim ($d_p$), hyperparams ($\alpha_1, \alpha_2, \alpha_3, \alpha_4$)}
\KwResult{Set of Predicted Transitions  $\mathcal{T}_i,\enspace
\forall \, d_i \in \mathcal{D}$}
{\small \tcp*[l]{\\ \textbf{Task Representation Learning} //}}
% \Begin{
    \ForEach{$d_i \in \mathcal{D}$}{
        $z_i \gets$ VisualFeatures $(d_i, w, d_p)$\;
        $k_i \gets$ KinematicFeatures $(d_i, w)$\;
        %  {\scriptsize  \tcp*[l]{concatenate kinematic and visual features}}
        $\mathbf{x}_i(t) \gets \binom{\mathbf{k_i}(t)}{\mathbf{z_i}(t)}\ \forall t \in \{1,\ldots, T_i\}$\;
        $\bar{x}_i(t) \gets \big[\, \mathbf{x}(t+1)^T, \ \mathbf{x}(t)^T, \ \mathbf{x}(t-1)^T \,\big]^T, \ \forall t$\;
        $N \gets \big[N^T, \bar{x}_i(1)^T,\ldots,\bar{x}_i(T_i)^T\big]^T$\;
    }
    {\scriptsize  \tcp*[l]{Cluster to get Change Points}}
    $\mathcal{C}^{CP} = \textnormal{DPGMM}(N, \alpha_0)$ {\scriptsize  \tcp*[r]{$\alpha_0$ is hyperparameter}}
    \ForEach{$N(n)\in\mathcal{C}^{CP}_{i},\ N(n+1)\in\mathcal{C}^{CP}_{j},i\neq j$ } {$CP \gets CP \cup \{N(n)\}$}
    {\scriptsize  \tcp*[l]{Cluster over Visual Feature Subspace}}
    $\mathcal{C}_{1} = \textnormal{DPGMM}(CP, \alpha_1)$ {\scriptsize  \tcp*[r]{$\mathcal{C}_{1}$: set of clusters}}
    %  \ForEach{$C^{1}_k \in \mathcal{C}^{1}$}{
    %     \uIf{$\sum_{d_i} \mathbf{1}\Big( \sum_{n:N(n)\in d_i} \mathbf{1}( CP(n) \in C^{1}_k) \geq 1 \Big) \geq \rho |\mathcal{D}|$}{
    %     $\mathcal{C}^{1}\gets \mathcal{C}^{1}\setminus \mathcal{C}^{1}_k$ {\scriptsize  \tcp*[r]{Cluster Pruning}}
    %     }
    %  }
    \ForEach{$C_{k} \in \mathcal{C}_{1}$}{
        $CP(C_{k}) \gets \{CP(n) \in C_{k}, \forall n \in \{1,\ldots,|CP|\}\}$\;
        {\scriptsize  \tcp*[l]{Cluster over Kinematic Feature Subspace}}
        $\mathcal{C}_{k2} \gets \textnormal{DPGMM}(CP(C_{k}), \alpha_2)$\;
        \ForEach{$C_{kk'} \in \mathcal{C}_{k2}$}{
            \If{$\sum_{d_i} \mathbf{1}\Big( \sum_{n:N(n)\in d_i} \mathbf{1}( CP(n) \in C_{kk'}) \geq 1 \Big) \leq \rho |\mathcal{D}|$}
            {$\mathcal{C}_{k2}\gets \mathcal{C}_{k2}\setminus \{C_{kk'}\}$ {\scriptsize \tcp*[r]{Cluster Pruning}}
            }
            {\scriptsize \tcp*[l]{collect intra-cluster transitions $ \forall \ d_i $}}
            $ \forall\ d_i \in \mathcal{D}$ \textbf{do} $T_i \gets T_i \cup \{CP(n) \in C_{kk'}, \forall n:N(n)\in d_i\}$\;
        }
    }
    {\small  \tcp*[l]{\textbf{\\ Temporal Segmentation of Each Demo} //}}
    \ForEach{$d_i \in \mathcal{D}$}{
        Repeat steps 1-17 for $\mathcal{D}' = \mathcal{D} \setminus d_i$\;
        $\mathtt{T}_j \gets \mathtt{T}_j \cup T_j^{(i)}, \ \{\forall\ j: d_j \in \mathcal{D}'\}${\scriptsize \tcp*[r]{$T_j^{(i)}$: \textit{i}th iteration}} 
        % $\mathcal{T}_i \gets DPGMM (\mathtt{T}, \alpha_4)${\scriptsize  \tcp*[r]{store cluster $\mu$ and $\Sigma$}}
    }
    {\scriptsize  \tcp*[l]{Cluster over time to predict Transition Windows}}
    \lForEach{$d_i \in \mathcal{D}$}{$\mathcal{T}_i \gets DPGMM (\mathtt{T}_i, \alpha_4)$}
    \KwRet{$\mathcal{T}_i,\enspace \forall \, d_i \in \mathcal{D}$}
% }
% \vspace*{-10pt}
\end{algorithm}
\setlength{\textfloatsep}{5pt}% Remove \textfloatsep



% \subsubsection{Visual Featurization}
% Once the images were pre-processed, we applied the convolutional filters from the pre-trained neural networks.
% Yosinski et al. note that CNNs trained on natural images exhibit roughly the same Gabor filters and color blobs on the first layer~\cite{yosinski2014NIPS}. 
% They established that earlier layers in the hierarchy give more general features while later layers give more specific ones. 
% In our experiments, we explore the level of generality of features required for segmentation. 
%Features drawn from layers 3 and 4 are generic as opposed to specific features from fully connected layers. 
% In particular, we explore two architectures designed for image classification task on natural images: (a) \textbf{AlexNet: } Krizhevsky et al. proposed multilayer (5 in all) a CNN architecture \cite{krizhevsky2012imagenet}, and (b)~\textbf{VGG: } Simoyan et al. proposed an alternative architecture termed VGG (acronym for Visual Geometry Group) which increased the number of convolutional layers significantly (16 in all)~\cite{simonyan2014very}.
% We also compare these features to other visual featurization techniques such as SIFT and SURF for the purpose of task segmentation using \TSC.

\vspace{2pt}
\noindent \textbf{Visual Feature Encoding and Dimensionality Reduction}: 

\subsubsection{Post-Processing: Encoding}
After constructing these features, the next step is encoding the results of the convolutional filter into a vector $z(t)$.
We explore three encoding techniques: (1) Raw values, (2) Vector of Locally Aggregated Descriptors (VLAD)~\cite{arandjelovic2013all}, and (3) Latent Concept Descriptors (LCD)~\cite{xu2014discriminative}.

\iffalse
\vspace{0.25em}
\noindent\textbf{Raw Filter Values: } The first encoding technique that we explore is using the raw convolutional filter values. We stack these values into a vector which constructs $z(t)$.

\vspace{0.25em}
\noindent\textbf{Vector of Locally Aggregated Descriptors (VLAD)\cite{arandjelovic2013all}: }
Vector of Locally Aggregated Descriptors (VLAD) image encoding as proposed by is a feature encoding and pooling method. 
It transforms an incoming variable-size set of independent samples into a fixed size vector representation.
VLAD encodes a set of local feature descriptors $I=\{x_1,\ldots,x_n\}$ extracted from an image using a code book $\mathrm{C} = \{c_1, \ldots, c_K \}$ built using a clustering method. With K-coarse cluster centers, generated by K-means, we can obtain the difference vector for every center $c_j$ as:\vspace{-5pt}
\[ u_k = \sum_{i:\,1\text{-}nbd(x_i)=\{c_j\}} (x_i- c_j)
\]
where $k\text{-}nbd(x_i)$ indicates k(=1) nearest neighbors of $x_i$ among K coarse centers.
VLAD vector is obtained by concatenating $u_k$ over all the K centers, $V = [u_1^T, \ldots, u_K^T]$, with $V$ of size K$D$ where $D$ is the dimension of incoming points $x_i$. 

\vspace{0.25em}
\noindent\textbf{Latent Concept Descriptors (LCD) \cite{xu2014discriminative}: }
Convolutional layers contain important spatial information, albeit at the cost of high dimensional representation. For instance the feature dimension at a convolutional layer can be $c\times c \times M$, where $c$ is filter size and $M$ is number of filters. However using the latent concept descriptors, a layer of size $c\times c \times M$ can be converted into $c^2$ latentwon. 
\fi

\subsubsection{Post-Processing: Dimensionality Reduction}
After encoding, we feed the CNN features $z(t)$, often in more than $50$K dimensions, through a dimensionality reduction process to boost computational efficiency. This also balances the visual feature space with a relatively small dimension of kinematic features ($<50$).
% This is for computational reasons as the visual features are very high dimensional ($>50K$) and \TS suffers from the curse of dimensionality.
Moreover, GMM-based clustering algorithms usually converge to a local minima and very high dimensional feature spaces can lead to numerical instability or inconsistent behavior.
We explore multiple dimensionality reduction techniques to find desirable properties of the dimensionality reduction that may improve segmentation performance.
In particular, we analyze Gaussian Random Projections (GRP), Principal Component Analysis (PCA) and Canonical Correlation Analysis (CCA) in Table~\ref{tab:visual}. GRP serves as a baseline, while PCA is used based on widely application in computer vision~\cite{xu2014discriminative}. We also explore CCA as it finds a projection that maximizes the visual features correlation with the kinematics. 


\subsection{Skill-Weighted Pruning} 
%Surgical demonstrations are evaluated by domain-experts using OSATS guidelines~\cite{niitsu2013OSATS}.
Demonstrators may have varying skill levels leading to increased outliers, and so we extend our outlier pruning to include weights.
Let, $w_i$ be the weight for each demonstration $d_i \in \mathcal{D}$, such that $w_i \in [0,1]$ and $\hat{w}_i = \frac{w_i}{\sum w_i}$. Then a cluster $C_{kk'}$ is pruned if it does not contain change points $CP(n)$ from at least $\rho$ fraction of demonstrations. This converts to:
\vspace{-3pt}
\[\sum_{d_i} \hat{w}_i\mathbf{1}\Big( \sum_{n:N(n)\in d_i} \mathbf{1}( CP(n) \in C_{kk'}) \geq 1 \Big) \leq \rho 
\]
\vspace{-5pt}

% the centroids of the transition state clusters.

% \vspace{0.5em}


% \subsubsection{Modeling Temporal Effects}
\vspace{3pt}
\noindent\textbf{Modeling Temporal Effects}\\
Time can be modeled as a separate sensing modality.
Without temporal localization, the transitions may be ambiguous.
For example, in a ``Figure 8" trajectory, the robot may pass over a point twice in the same task.
We define an augmented state space $\mathbf{x}(t) = \binom{k(t)}{t}$.
Within a state cluster, we model the times which change points occur as drawn from a GMM: $ t \sim N(\mu_i, \sigma_i)$, 
then we can apply a GMM to the set of times.
This groups together events that happen at similar times during the demonstrations.
The result is clusters of states and times.
Thus, a transition state $m_k$ is defined as tuple of an ellipsoidal region of the state-space and a time interval.

%\subsection{Other Extensions}
\subsection{Sliding Window States}
To better capture hysteresis and transitions that are not instantaneous, in this current paper, we use rolling window states where each state $\mathbf{x}_{(t)}$ is a concatenation of $T$ historical states.
We varied the length of temporal history $T$ and evaluated performance of the \TSC algorithm for the suturing task using metric defined in Section~\ref{sec:metrics}.
We empirically found a sliding window of size 3, i.e.,  $\mathbf{x}_{(t)} = \binom{\mathbf{k}(t)}{\mathbf{z}(t)}$, as the state representation led to improved segmentation accuracy while balancing computational effort. 

\subsection{Robust Temporal Clustering}
To reduce over-fitting and build a confidence interval as a measure of accuracy over the temporal localization of transitions, we use a Jackknife estimate. It is calculated by aggregating the estimates of each $N-1$ estimate in the sample of size $N$.
We iteratively hold out one out of the $N$ demonstrations and apply \tsc to the remaining demonstrations. Then, over $N-1$ runs of \tsc, $N-1$ predictions are made $\forall d_i \in \mathcal{D}$. We temporally cluster the transitions across $N-1$ predictions, to estimate final transition time mean and variance $\forall d_i \in \mathcal{D}$. This step is illustrated in step 20-21 of Algorithm~\ref{alg:vtsc}.

%\todo{describe the jack-knife estimate to get confidence intervals.}


\end{document}
\section{Visual Featurization}

\todo{fill in}
\begin{enumerate}
\item Describe goals

\begin{enumerate}
\item Spatially \& Scale invariant features
\item Key in on primitives
\end{enumerate}

\item Describe different ways that you can get visual features
\begin{enumerate}
\item Deep features from Convolutional Neural Networks
\item Pre-trained models from CAFFE - vanilla AlexNet (Krizevksy et al.)
\item Pre-trained models from CAFFE - VGG (Simonyan and Zisserman)
\item Encoding (Doesn't really work now)
\item SIFT/SURF (Lowe)
\item Dense Trajectories (Heng Wang)
\end{enumerate}

\item Describe what we did and our methodology

\item Pre-processing
\begin{itemize}
\item Background subtraction
\begin{itemize}
\item A OpenCV built-in background subtraction algorithm was applied on each frame before pushing through the CNN. The algorithm was a Gaussian Mixture-based Background/Foreground Segmentation algorithm. It was introduced in the paper "An improved adaptive background mixture model for real-time tracking with shadow detection" by P. KadewTraKuPong and R. Bowden in 2001.
\end{itemize}

\item Cropping \& Scaling
\begin{itemize}
\item I cropped all frames by equal amounts to capture only the workspace where most of the robot manipulation happened. Then, they were rescaled to 640 x 480 dimensions. All pre-processing happened with ffmpeg.
\end{itemize}

\end{itemize}


\end{enumerate}
\section{Latent State $H_t$}
Describe the process of linking kinematics and video (PCA, CCA, etc.).
If we apply any VLAD or encoding, or batching, describe it here:

\subsection{Temporal Batching}
We have so far used the concatenation of kinematic and visual features as the state representation $\binom{k(t)}{z(t)}$. However, the use of rolling time window in both kinematics and visual space allows capturing dynamics. Explicit use of first and second numerical derivatives in state representation for learning complex dynamics has been the state-of the art\tocite. 

We propose the use of a sequence raw states as the representation. Addition of every time step in the state implicitly corresponds to adding derivatives. However, the benefit of adding more time history saturates while the computational intensity requires scales quickly, especially with the use of high dimensional image features. 

We propose the use of $[k(t-1), k(t), k(t+1)]^T$ as the kinematic state representation. Other lengths of temporal history were experimented with as shown in Figure~\todo{add figure}. We note that while we use a 3 step temporal history, with data being captured at 30Hz, the amount of movement may be negligible between consecutive frames. We sub-sample the kinematic data to 10Hz to magnify kinematic changes. However, we keep the video data at 30Hz and use 9 consecutive frames for each $z(t)$. Using a CNN based featurization results in a very high dimensional feature vector, while use of SIFT like features results in a a feature vector of different lengths across different frames. Hence an encoding procedure as described following in employed to recover a concise and fixed dimensional representation for video subsequences. 

\todo{Current Status}: While batching is supposed to help in video analysis , I've seen good separation of clusters (PCA on conv features) with just individual frames without clustering... I think We need to look into this more and need to test it out with milestones clustering.


\subsection{Video Encoding}
Vector of Locally Aggregated Descriptors (VLAD) image encoding as proposed by \cite{arandjelovic2013all}\ignore{\cite{jegou2010aggregating}} is a feature encoding and pooling method, similar to Fisher vectors. 
It transforms an incoming variable-size set of independent samples into a fixed size vector representation.
VLAD was recently shown to perform better than Fisher vectors and average pooling for encoding multiple frames~\cite{xu2014discriminative}.

VLAD encodes a set of local feature descriptors $I=\{x_1,\ldots,x_n\}$ extracted from an image using a code book $\mathrm{C} = \{c_1, \ldots, c_K \}$ built using a clustering method such as Gaussian Mixture Models (GMM) or K-means clustering. With K-coarse cluster centers, generated by K-means, we can obtain the difference vector for every center $c_j$ as:
\[ u_k = \sum_{i:\,1\text{-}nbd(x_i)=\{c_j\}} (x_i- c_j)
\]
where $k\text{-}nbd(x_i)$ indicates k(=1) nearest neighbors of $x_i$ among K coarse centers.
VLAD vector is obtained by concatenating $u_k$ over all the K centers, $V = [u_1^T, \ldots, u_K^T]$, with $V$ of size K$D$ where $D$ is the dimension of incoming points $x_i$. 

Further, \cite{kantorov2014efficient} showed that VLAD-k, an extension to k-nearest neighbors outperforms vanilla VLAD (k=1). VLAD vectors are generally normalized using power normalization ($sign(u_k)\sqrt{\|u_k\|_2}$) followed by $L_2$ normalization of $V$. However an intra-normalization has been shown to perform better in balancing all the features~\cite{arandjelovic2013all}. This entails: $\hat{u}_k = u_k/\|u_k\|_2$ followed by $L_2$ normalization of $V$.
We use VLAD-k with k=5 followed by intra-normalization.

\subsection*{Latent Concept Descriptors}

It is worth noting that convolutional filters can be regarded as generalized linear classifiers on spatial patches, and each conv filter can be equated to a latent concept as proposed in~\cite{xu2014discriminative}. We represent the features from convolutional layers as vectors of latent concepts, with each entry representing a response to a latent concept. Since, conv filters at every layer are independent, each filter response can be understood as the prediction on linear classifier on respective latent concept.

Convolutional layers contain spatial information, however at the cost of high dimensional representation. For instance the feature dimension at a convolutional layer can be $c\times c \times M$, where $c$ is filter size and $M$ is number of conv filters. Simply flattening the layer would result in a high dimensional representation inducing computational instability. Since convolutional layers are higher dimensional than fully connected layers, they are often not used directly.

\todo{re-phrase}
However using the latent concept descriptors, a conv layer of size $c\times c \times M$ can be converted into $c^2$ latent concept descriptors with M dimensions. Each latent concept descriptor repre- sents the responses from the M filters for a specific pool- ing location. Once we obtain the latent concept descriptors for all the frames in a video, we then apply an encoding method to generate the video representation. In this case, each frame contains a2 descriptors instead of one descriptor for the frame.

\todo{Current Status}: We have also implemented the following encoding method- Latent Content Descriptors (LCD)+ VLAD \cite{xu2014discriminative}. However, initial results weren't super but we need to see how it performs on milestone clustering.

\subsection{Encoding Dimensionality}
The output of the encoding step is of the dimension $DK$, with D being the dimension of incoming points. Particularly in case of CNN features drawn from convolutional layers can run into tens of thousands leading to imbalance in the vectors after concatenation with kinematics. 

Hence 



\section{Results}
\subsection{Exp1. End-to-end result with some task}

\begin{enumerate}
\item Show that clusters are sensible and align with some intuitive criteria e.g., surgemes
\end{enumerate}

\subsection{Exp2. Does Vision Help}

\begin{enumerate}
\item Remove visual features and show that clusters degrade
\end{enumerate}

\subsection{Exp3. Parameter Search}

\begin{enumerate}
\item Describe our eval procedure and how we arrived at the architecture we did.
\end{enumerate}

\subsection{Exp4. Robustness}
\begin{enumerate}
\item Add noise or corrupt images and test to see how robust the segmentations we learn are.
\end{enumerate}


\subsection{Discussion}
\begin{enumerate}
\item How successful was our unsupervised approach in learning meaningful segmentations
\item RGB videos vs. RGB-D videos
\end{enumerate}


\input{8-conclusion.tex}

\subsubsection*{Acknowledgement}
This work is supported in part by a seed grant from the UC Berkeley CITRIS, and by the U.S.\ NSF Award IIS-1227536: Multilateral Manipulation by Human-Robot Collaborative Systems. We thank Intuitive Surgical, Simon DiMao, and the dVRK community for support; NVIDIA for computing equipment grants; Andy Chou and Susan Lim for developmental grants; and Sergey Levine and Katerina Fragkiadaki.


\bibliographystyle{IEEEtranS}
\bibliography{deepP2P}

\end{document}
