\documentclass[letterpaper, 10 pt, conference]{ieeeconf}


%\let\labelindent\relax
\input{preamble}
\usepackage{blindtext}

\newboolean{include-notes}
\setboolean{include-notes}{true}
\newcommand{\fp}[1]{\ifthenelse{\boolean{include-notes}}%
 {\textcolor{blue}{\textbf{FP: #1}}}{}}
%===============================================================
\title{\LARGE \bf
Learning Sub-Task Level Semantic Segmentation of Multi-Step Task Trajectories from Video with Deep Learning
\\{\color{blue} V0.2 Wed 09-02-2015 11:05am}
}
% Pixels to Primitives (P2P)

\author{%
Adithyavairavan Murali*, Animesh Garg*, Sanjay Krishnan*, Florian Pokorny, Pieter Abbeel, Ken Goldberg
\thanks{\hrule \vspace{5pt} * The authors contributed equally to the paper}%
\thanks{EECS \& IEOR, University of California, Berkeley CA USA; \texttt{\{adithya\_murali, animesh.garg, sanjaykrishnan, ftpokorny, goldberg\}@berkeley.edu}}%
% \thanks{$^{1}$EECS, University of California, Berkeley; {\{sanjaykrishnan, adithya\_murali\}@berkeley.edu}}%
% \thanks{$^{2}$IEOR and EECS, University of California, Berkeley; {\{animesh.garg, goldberg\}@berkeley.edu}}%
}
\IEEEoverridecommandlockouts %to enable thanks to appear
\newcommand{\sys}{\textsf{TSC+VIS}\xspace}

\begin{document}

\maketitle

\begin{abstract}
    
Segmentation 
\fp{let's disambiguate: temporal segmentation?}
is an important first step in analyzing long running robotic tasks. For reliable results, it is important \fp{too strong
    assertion - incorporate Ken's feedback on this. I.e. relationship to previous work, factual statement of what we
investigate } to consider both visual and kinematic data, as visual data provides important 
information about the state of the workspace. \fp{avoid such statements - we investigate this} 
Existing unsupervised segmentation methodologies are limited in the ways they can leverage visual data and rely on annotations or complete knowledge of all objects in the world. In this paper, we propose a framework that takes a step towards unsupervised segmentation of robotic demonstrations using raw video (i.e., pixel data). 
We identify key transition events in kinematic data, cluster transitions together using visual data, and then identify segments of the raw video corresponding to these clusters of transition events. 
The resulting video segments can be used to design error recovery actions, parameter tuning, action classification, 
and operator skill assessment.\fp{overstatement, instead list experimental contributions only and that we extend your
ISRR work} 
\todo{Our results on x suggest y}
\end{abstract} 

\documentclass[0-main.tex]{subfiles}
\begin{document}


\section{Introduction}
%A large and ever increasing number of surgical procedures are performed or aided by surgical robots such as Intuitive Surgical's \textit{da Vinci}.
The prevalence of robot-assisted surgery is leading to a proliferation of datasets consisting of kinematic and fixed-camera video recordings of surgical procedures.
While these datasets have the potential to facilitate learning and autonomy, the variability of surgical data poses a unique challenge.
By nature, surgical robots interact with deformable environments and varying anatomy. 
Therefore, one of the key challenges in utilizing these datasets is to learn a common sequential structure shared across multiple instances of the same surgical task.
One model for this sequential task structure is the Transition State model~\cite{krishnan2015tsc}, which identifies spatially and temporally similar changes in motion.
Thus, while the apparent data might be highly variable, the model often learns latent similarities in tool position or kinematic state.
The learned model temporally \emph{segments} trajectories into meaningful contiguous sections, which can ultimately facilitate local learning from demonstrations, skill assessment, and salvaging good local segments from otherwise inconsistent demonstrations.

In addition to the Transition State model, there are several recent proposals to learn segmentation criteria directly from data with minimal supervision~\cite{calinon2010learning, Niekum2015learning}.
Inherently, the success of these approaches is dependent on the state representation, which is particularly challenging for visual features; especially when we want these features to perform well across different tasks. 
Inspired by the recent success of deep neural networks in reinforcement learning~\cite{levine2015end, lenz2015deep}, this paper explores how segmentation can be learned from visual state representations extracted from \textit{deep} convolutional neural networks (CNNs).

\begin{figure}[t!]
\centering
\vspace{-5pt}
% \includegraphics[width=0.5\linewidth]{figures/insert}
\includegraphics[width=\linewidth]{figures/suturing-teaser-v4.png}
\caption{We apply \tsc to a suturing task. Each ``throw" of suturing repeats between four steps, and figure illustrates that \tsc extracts a segmentation that closely aligns with the manual annotation without supervision.}
\figlabel{surgical-teaser}
% \label{fig:pr2_toyplane}
\vspace{-15pt} 
\end{figure}
We propose Transition State Clustering with Deep Learning (\tsc), which extends our previous work~\cite{krishnan2015tsc} with automatically constructed visual features using pre-trained CNNs.
In our prior work, we identified changes in local linearity in each trajectory, and learned a model to infer regions of the state-space at which changes occurred.
We modeled these regions as generated from a hierarchical nonparametric Bayesian model, where the number of regions are determined by a Dirichlet Process and the shape of the regions are determined by a mixture of multivariate Gaussian random variables.
Our surprising finding is that while the original Transition State model was motivated for spatial states, it empirically performs well (in comparison to ground truth) even when applied to trajectories of visual filters derived from layers of CNNs.

However, the integration of the convolutional filters with kinematics is not as straight-forward as concatenating the two parameter spaces.
This paper addresses two new issues: (1) defining metrics across multiple sensing modalities, and (2) task-trajectory correspondence.
For (1), it can be challenging to define similarity metrics across states that have both spatial and visual components.
We propose a hierarchical clustering approach that first clusters over visual space and then over kinematics, which allows us to compare kinematics-kinematics and visual-visual.  
For (2), \tsc learns a segmentation model for a task given observation of trajectories of the task.
The problem is that we may want to know how to correspond states from the individual trajectories to the segments of the task learned across all trajectories.
To learn this correspondence, we propose a novel temporal clustering technique which leverages the Jackknife estimator to correspond global segmentation structure to temporal points in the individual trajectories.




 \iffalse
\begin{SCfigure*}[][t!]
    \centering
    \vspace{-10pt}
    \includegraphics[width=1.55\linewidth]{figures/sysArch}
    \caption{We use a visual processing pipeline with deep features to construct a trajectory of high-dimensional visual states $z(t)$.
    We concatenate encoded versions of these features with kinematics and apply hierarchical clustering to find segments.}%dont add more lines--seems to mess up formatting
    \label{fig:pipeline}
    \vspace{-10pt}
\end{SCfigure*}
\fi
% \fi

% \begin{figure*}[!t]
% \centering
% \includegraphics[width=0.8\linewidth]{figures/sysArch}
% \caption{\tsc is a task segmentation pipeline that starts with multimodal demonstrations (kinematics and vision), uses deep learning to featurize the frames of the video, and then clusters transition events to find segments.}
% \label{fig:pipeline}
% \vspace{-15pt}
% \end{figure*}


We evaluate \tsc on three datasets: (1) a synthetic 4 segment example, (2) JIGSAWS surgical needle passing,  and (3) JIGSAWS surgical suturing.
On the synthetic example, we find that \tsc recovers the 4 underlying segments in the presence of partial state observation (one kinematic state hidden), control noise, and sensor noise. 
We also compare \tsc with manual annotations when available.
On real datasets, we find that \tsc matches the manual annotation with up to 0.806 normalized mutual information.
Our results also suggest that including kinematics and vision results in increases of up-to 0.215 NMI over kinematics alone.
We demonstrate the benefits of using an unsupervised approach by presenting examples where \tsc discovers unlabeled segments due to human annotator error (as shown in Figure \ref{fig:suturing}), and can learn across demonstrations with widely varying operator skill levels (as shown in Table \ref{tab:jigsaws}).


%Clustering allows focussed LFD on each segment resulting in a better fit. 
%Clustering can identify inconsistent outliers and dirty data segments that cann corrupt LFD, also 
%Can identify segments where more consistent data is needed.
%can salvage good segments form bad demo sequences. 

%emph unsupervised
%avoids human bias
%apply consistent models -- can discover subtle transitions
%faster than humans 
  
\iffalse
\begin{figure}[ht]
\centering
\includegraphics[width=\columnwidth]{figures/architecture.pdf}
\caption{\todo{name} architecture. We use a pre-trained CNN to featurize raw video data for use in segmentation. After featurization, we combine the data with kinematic data and apply a Transition State Clustering algorithm to identify segments.}
\figlabel{arch}
\vspace{-1em}
\end{figure}
\fi

\end{document}
\documentclass[0-main.tex]{subfiles}
\begin{document}

\section{Related Work}

%\subsection{Related Work}

\subsection{Learning From Demonstrations}
One model for learning from demonstrations uses segmentation to discretize action spaces (skill-learning) which allows for efficient learning of complex tasks~\cite{ijspreet2002learning,pastor2009learning}.
This line of work largely focuses on pre-defined primitives.
Niekum et al. \cite{niekum2012learning} proposed an unsupervised extension to the motion primitive model by learning a set of primitives using the Beta-Process Autoregressive Hidden Markov Model (BP-AR-HMM).
The work by Niekum et al. does incorporate visual information, however, it does not use visual information to actually find segments.
Post segmentation,  Niekum et al. uses AR markers to estimate poses of all of the objects in the workspace.
The segments, discovered with kinematics alone, are then specified in each objects reference frame.
When the objects are then moved, the trajectory can be transferred using a Dynamic Motion Primitive model.

Calinon et al.~\cite{calinon2014skills, calinon2010evaluation} characterizes segments from demonstrations as skills that can be used to parametrize imitation learning.
In this line work, the authors apply Gaussian Mixture Models (GMMs) to cluster observations from the same mixture component.
A number of other works have leveraged this model for segmentation e.g., \cite{konidaris2009efficient, konidaris2011robot, subramanian2011learning}.
As we will later describe, Gaussian Mixture Models have a duality with switched linear dynamical systems \cite{moldovan2013dirichlet}.
Calinon et al. \cite{calinon2010evaluation} uses segmentation to teach a robot how to hit a moving ball.
They use visual features through a visual trajectory tracking of a ball.
The visual sensing model in Calinon et al. is tailored to the ball task, and in this paper, we use a set of general visual features for all tasks.

\subsection{Surgical Robotics}
%Surgical robotics has a number of challenges that make segmentation difficult \cite{krishnan2015tsc}.
%In Krishnan et al. \cite{krishnan2015tsc}, we found that hand-annotated visual features could significantly improve segmentation accuracy.
Other surgical robotics works have largely studied the problem of supervised segmentation using either segmented examples or a pre-defined dictionary of motions (similar to motion primitives) \cite{varadarajan2009data,tao2013surgical,lea15improved, zappella2013surgical, quellec2014segmentation}.


\subsection{Visual Gesture Recognition}
A number of recent works, attempt to segment human motions from videos \cite{hoai2011joint, tang2012learning, yang2013discovering, jones2014unsupervised, wu2014leveraging, wu2015watch}.
Tang et al. and Hoai et al. proposed supervised models for human action segmentation from video.
Building on the supervised models, there are a few unsuperivsed models for segmentation of human actions: Jones et al.\cite{jones2014unsupervised}, Yang et al. \cite{yang2013discovering}, Di Wu et al. \cite{wu2014leveraging} , and Chenxia Wu et al. \cite{wu2015watch}.
Jones et al. \cite{jones2014unsupervised} restricts their segmentation to learning from two views of the dataset (i.e., two demonstrations).
Yang et al. \cite{yang2013discovering} and Wu et al.  \cite{wu2015watch} use k-means to learn a dictionary of primitive motions, however, in prior work, we found that transition state clustering outperforms a standard k-means segmentation approach.
In fact, the model that we propose is complementary to these works and would be a robust drop-in-replacement for the k-means dictionary learning step \cite{krishnan2015tsc}.
The approach taken by Di Wu et al. is to parametrize human actions using a skeleton model, and they learn the parameters to this skeleton model using a deep neural network.
In this work, we explore using generic deep visual features for robotic segmentation without requiring task-specific optimization such as skeleton or action models using in human action recognition.

\subsection{Deep Features in Robotics}
Robotics is increasingly using deep features for visual sensing. For example, Lenz et al. uses pre-trained neural networks for object detection in grasping \cite{lenz2015deep} and
Levine et al. \cite{levine2015end} fine-tune pre-trained CNNs for policy learning.
For this reason, we decide to explore methodologies for using deep features in segmentation as well. We believe that segmentation is an important first step in a number of robot learning applications, and the appropriate choice of visual features is key to accurate segmentation.
We present an initial exploration of different visual featurization strategies and segmentation accuracy.

\iffalse
\subsection{Trajectory Segmentation Models}
Many unsupervised segmentation models either implicitly or explicitly model the dynamics as locally linear.
It is important to note that locally linear dynamics does not imply linear motions, as spiraling motions can be represented as linear systems. 
In \cite{elhamifar2009sparse}, videos are modeled as transitions on a lower-dimensional linear subspace and segments are defined as changes in these subspaces.
Willsky et al~\cite{willsky2009sharing} proposed BP-AR-HMM, which was applied by Niekum et al. in robotics \cite{niekum2012learning}.
This model is explicitly linear by fitting a autoregressive model to time-series. 
The linear function switches according to an HMM with states parametrized by a Beta-Bernoulli model (i.e., Beta Process). 
In fact, even the works that apply Gaussian Mixture Models for segmentation \cite{calinon2010learning, lee2015autonomous, kruger2012imitation}, implicitly fit a locally linear dynamical model.
Moldovan et al. \cite{moldovan2013dirichlet} proves that a Mixture of Gaussians model is equivalent to Bayesian Linear Regression; i.e., when applied to a time window it fits a linear transition between the states.
\fi

\end{document}
\section{Prior Work: Transition State Clustering}

We first overview the Transition State Clustering algorithm from our prior work \cite{krishnan2015tsc}.

\subsection{Transition State Clustering Overview}
The Transition State Clustering algorithm (\sys), learns clusters of states that mark dynamical regime transitions.

\subsubsection{Learning Transition States}
Let $\mathcal{D}=\{d_i\}$ be the set of demonstrations where each $d_i$ is a trajectory of fully observed robot states and each state is a vector in $\mathbb{R}^d$.
We model each demonstration as a switched linear dynamical system.
There is a finite set of $d \times d$ matrices $\{A_1,...,A_k\}$, and an i.i.d zero-mean additive Gaussian Markovian noise process $W(t)$ which accounts for noise in the dynamical model:
\[
\mathbf{x}(t+1) = A_{i}\mathbf{x}(t) + W(t) \text{ : } A_i \in \{A_1,...,A_k\}
\]
In this model, transitions between regimes are instantaneous where each time $t$ is associated with exactly one dynamical system matrix $1,...,k$.
\emph{Transition states} are defined as the last states before a dynamical regime transition in each demonstration.
Therefore, there will be times $t$ at which $A(t) \ne A(t+1)$.
A transition state is the state $x(t)$ at time $t$.

Suppose there was only one regime, then this would be a linear regression problem:
\[
\arg\min_A \|A X_t - X_{t+1}\|
\]
where $X_t$ is a matrix where each column vector is $x(t)$, and $X_{t+1}$ is a matrix where each column vector is the corresponding $x(t+1)$.
Moldovan et al. \cite{moldovan2013dirichlet} proves that fitting a Jointly Gaussian model to $n(t) = \binom{\mathbf{x}(t+1)}{\mathbf{x}(t)}$ is equivalent to Bayesian Linear Regression.
We use Dirichlet Process Gaussian Mixture Models (DP-GMM) to learn the regimes without have to set the number of regimes in advance.
Each cluster learned signifies a different regime, and co-linear states are in the same cluster.
To find transition states, we move along a trajectory from $t=1,...,t_f$, and find states at which $n(t)$ is in a different cluster than $n(t+1)$.
These points mark a transition between clusters (i.e., transition regimes).

\subsubsection{Learning Transition State Clusters}
A \emph{transition state cluster} is
defined as a clustering of the set of transition states across all demonstrations; partitioning these transition states into $m$ non-overlapping similar groups:
$
\mathcal{C} = \{C_1, C_2,...,C_m\}
$
In our prior work, we formalized the conditions under which meaningful clusters can be learned (i.e., consistency of demonstrations \cite{krishnan2015tsc}).
If we model the states at transition states as drawn from a GMM model: ${x}(t) \sim N(\mu_i, \Sigma_i)$.
Then, we can apply the DP-GMM again to cluster the state vectors at the transition states.
Each cluster defines an ellipsoidal region of the state-space space.

Each of these clusters will have constituent vectors where each $n(t)$ belongs to a demonstration $d_i$. 
Clusters whose constituent vectors come from fewer than a fraction $\rho$ demonstrations are \emph{pruned}.
$\rho$ should be set based on the expected rarity of outliers.

\vspace{0.5em}

\noindent \textbf{Given a consistent set of demonstrations, the algorithm finds a sequence of transition state clusters reached by at least a fraction $\rho$ of the demonstrations.}

\input{4-cnn.tex}
\section{Latent State $H_t$}
Describe the process of linking kinematics and video (PCA, CCA, etc.).
If we apply any VLAD or encoding, or batching, describe it here:

\begin{itemize}
\item Encoding - Current Status
\begin{itemize}
\item So I've implemented the following encoding method- Latent Content Descriptors (LCD) + VLAD. However, initial results weren't great and I need to see how it performs on milestones clustering.
\end{itemize}


Vector of Locally Aggregated Descriptors (VLAD) image encoding as proposed by \cite{jegou2010aggregating} is a method a feature encoding and pooling method, similar to Fisher vectors. VLAD encodes a set of local feature descriptors $I=(x_1,\ldots,x_n)$ extracted from an image using a codebook $\mathrm{C} = \{c_1, \ldots, \c_m \}$ built using a clustering method such as Gaussian Mixture Models (GMM) or K-means clustering.


VLAD was shown to perform better than Fisher vectors and average pooling for encoding multiple frames~\cite{xu2014discriminative}.

\item Temporal Batching - Current Status
\begin{itemize}
\item While batching is supposed to help in video analysis , I've seen good separation of clusters (PCA on conv features) with just individual frames without clustering... I think I need to look into this more and need to test it out with milestones clustering.
\end{itemize}

\end{itemize}

\section{Clustering}
Describe the clustering procedure--refer to ISRR when needed.

\section{Results}
\subsection{Exp1. End-to-end result with some task}

\begin{enumerate}
\item Show that clusters are sensible and align with some intuitive criteria e.g., surgemes
\end{enumerate}

\subsection{Exp2. Does Vision Help}

\begin{enumerate}
\item Remove visual features and show that clusters degrade
\end{enumerate}

\subsection{Exp3. Parameter Search}

\begin{enumerate}
\item Describe our eval procedure and how we arrived at the architecture we did.
\end{enumerate}

\subsection{Exp4. Robustness}
\begin{enumerate}
\item Add noise or corrupt images and test to see how robust the segmentations we learn are.
\end{enumerate}


\subsection{Discussion}
\begin{enumerate}
\item How successful was our unsupervised approach in learning meaningful segmentations
\item RGB videos vs. RGB-D videos
\end{enumerate}


\input{8-conclusion.tex}

\subsection*{Acknowledgement}


\bibliographystyle{IEEEtranS}
\bibliography{deepP2P}

\end{document}
